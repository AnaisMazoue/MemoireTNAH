\chapter*{Conclusion}
	\addcontentsline{toc}{chapter}{Conclusion}
	Ce mémoire s'est efforcé de décrire le travail réalisé au cours du stage effectué au sein de l'équipe du projet de recherche \acrshort{alfa}, à l'Observatoire de Paris. Le principale mission était la scénographie d'une exposition virtuelle d'un corpus de manuscrits issus de l'astronomie alphonsine, courant scientifique européen qui se développe entre les XIIIe et XVIe siècles. Après avoir développé la plateforme \acrshort{dishas} pour valoriser les données produites dans le cadre de ce projet de recherche et conçu des outils comme le \textit{survey} ou \textit{Table transcriber}, l'équipe avait à cœur de mettre en scène une exposition. Les humanités numériques ayant une place de choix dans le projet \acrshort{alfa}, se tourner vers une exposition en ligne faisait sens, d'autant plus que la plupart des manuscrits sélectionnés pour en faire partie sont disponibles sous un format numérique. \\
	
	Plusieurs étapes se sont succédées pour aboutir à la création d'une plateforme d'exposition virtuelle accessible au public : une phase d'analyse, une phase de rédaction, une phase d'exécution, une phase de documentation et une phase de mise en production. \\
	
	La première partie du présent mémoire se rapporte ainsi plutôt aux premières phases d'analyse et de rédaction. En effet, après avoir pris connaissance du contexte et des principaux enjeux du projet, un important travail de reprise des données produites par les chercheurs a été effectué en vue de leur harmonisation et de leur réutilisation. C'est dans ce même temps qu'on pu être listées les exigences en matière de contenu sur le futur site web et qu'une réflexion a pu être entamée sur la définition à donner à l'objet d'exposition virtuelle. 
	
	La seconde partie touche, quant à elle, à la scénographie de façon concrète. En effet, les enjeux liés à l'expérience des visiteurs sont essentiels : la navigation au sein des différents parcours doit permettre un accès au contenu du corpus alphonsin. C'est aussi l'occasion de réfléchir au rôle des ingénieurs numériques dans un projet de recherche qui fait dialoguer considérations scientifiques et contraintes techniques et à leur mission de médiation entre les deux. Du fait de cette mission, la rédaction de documentation à destinations des chercheurs comme des ingénieurs a également occupé une place importante au cours de ce stage.
	
	Enfin, le descriptif et l'analyse du développement de l'interface viennent clore ce mémoire. L'utilisation du protocole \acrshort{iiif} a tenue une place de premier plan pour la mise en avant des numérisations. De plus, développer une plateforme \textit{responsive} et réutilisable était au centre des préoccupations afin de pouvoir concilier les besoins des utilisateurs et la recherche \textit{open source}. La proprification du code et des données est un moyen de mettre en valeur le travail des chercheurs, qui a lui-même pour objectif de dévoiler aux yeux d'un public plus large le corpus alphonsin. \\
	
	Ce stage a été une expérience très enrichissante du fait des connaissances acquises d'un point de vue technique mais également pour la découverte de ce corpus d'histoire de l'astronomie. De plus, avoir la possibilité d'évoluer dans un projet international est formateur pour aborder un milieu où l'anglais fait partie du bagage requis. \\
	
	La médiation numérique offre donc des possibilités grandissantes qui permettent de s'affranchir des barrières spatiales pour la réalisation d'une exposition. En effet, les œuvres mises en scène peuvent venir des quatre coins du monde et rester à l'abri dans leur lieu de conservation. De même, les visiteurs peuvent, depuis chez eux, accéder à de riches contenus. Malgré les avancées réalisées au cours de ce stage, de nombreuses possibilités restent à explorer pour parfaire la mise en lumière de ces manuscrits qui révèlent la richesse de l'astronomie alphonsine. 
