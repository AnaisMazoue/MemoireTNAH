\chapter{Introduction}

\og{}La philosophie \footnote{Par philosophie, on entend alors science des principes et des causes, si l'on suit la définition fournie par Godefroy dans son \textit{Dictionnaire de l'ancienne langue française et de tous ses dialectes du 9e au 15e siècle}.} est écrite dans ce grand livre, l'univers, continuellement ouvert devant nos yeux, mais elle ne peut être comprise à moins d'en apprendre d'abord la langue et d'en connaître les caractères. Elle est écrite en langage mathématique et les caractères sont des triangles, des cercles et autres figures géométriques sans lesquelles il est humainement impossible d'en comprendre le moindre mot. \fg{}\footnote{Cette traduction est une proposition personnelle de l'originale en italien : \og{}\textit{La filosofia è scritta in questo grandissimo libro che continuamente ci sta aperto innanzi a gli occhi (io dico l'universo), ma non si puo intendere se prima non s'impara a intender la lingua, e conoscer i caretteri, ne' quali è scritto. Egli è scritto in lingua matematica, e i caratteri son triangoli, cerchi, ed altre figure geometriche, senza i quali mezi è impossibile a intenderne umanamente parola}. \fg{} Ces quelques mots ont été recueillis au \textit{Museo Galileo : Istituto e Museo di Storia della Scienza} de Florence et sont issus de l'ouvrage suivant : \textit{Il Saggiatore nel quale con bilancia esquisita et giusta si ponderano le cose contenute nella libra astronomica et filosofica di Lotario Sarsi, etc.}.}. Avec ces mots, en 1623, Galilée \footnote{{Mathématicien, astronome, physicien et géomètre né à Pise en 1564 et mort à Arcetri en 1642, voir} \cite{baldiniGalileoGalilei1998}.} place les mathématiques au centre de la compréhension de l'univers. Si les travaux de Galileo Galilei sont postérieurs à la période de l'astronomie alphonsine qui est au cœur du présent mémoire, ils en sont héritiers. En effet, l'utilisation d'outils mathématiques pour tenter de comprendre le monde n'est pas une nouveauté du XVIIe siècle, bien que cette période marque des changements majeurs dans l'histoire des sciences et de l'astronomie en particulier. 

C'est aussi au XVIIe siècle que sont construits des observatoires astronomiques comme ceux de Paris ou de Greenwich. L'Observatoire de Paris, fondé en 1667, est le plus ancien encore en activité aujourd'hui et c'est dans cette institution que j'ai été accueillie en tant que stagiaire ingénieure numérique au sein de l'équipe d'histoire des sciences, du 28 mars au 4 août 2022. Cette équipe est composée de sept personnes sur site à Paris : Matthieu Husson, chef du projet \acrshort{alfa}, Ségolène Albouy, cheffe de projet numérique, Eleonora Andriani, Sophie Serra, Scott Trigg, chercheurs post-doctorants, Camille Bui, doctorante,  et Tristan Dot, ingénieur. Toutefois, ce sont des dizaines de personnes qui prennent en réalité part à ce projet et l'équipe complète est dispersée dans de nombreux pays. C'est l'une des richesses du projet \acrshort{alfa} que d'être en mesure de mobiliser ainsi une équipe véritablement internationale. De ce fait, les publications et réalisations se font en anglais. Mon stage n'a pas dérogé à cela et c'est ce qui justifie que les documents présentés, notamment en annexes, soient rédigés en anglais. 

Si les chercheurs s'intéressent à des sources diverses mettant chacune en avant des aspects aussi différents que complémentaires de l'histoire de l'astronomie, l'une des ambitions de \acrshort{dishas} (\acrlong{dishas}) est de mettre au point des outils communs pour l'étude de ces ressources variées. Pour la conception de ces outils, des ingénieurs ont été missionnés : l'équipe de recherche est accompagnée d'une équipe chargée de projets numériques, le projet est bel et bien à inclure dans le champs des \acrlong{dh} ou humanités numériques. Cette notion renvoie au contact, de plus en plus fréquent et apprécié, entre les technologies numériques et les sciences humaines et sociales \footnote{\og{}Le terme \og{}humanités\fg{} recouvre ici l’ensemble des sciences humaines et sociales (SHS) et les patrimoines et corpus qu’elles traitent. Le terme \og{}numérique\fg{} renvoie à l’ensemble des procédés et techniques qui permettent de transformer n’importe quel objet en ensemble de données binaires, les algorithmes informatiques qui traitent ces données (y compris les conserver et en prendre soin) ainsi que les procédés qui génèrent des rendus tangibles des résultats obtenus, notamment sous forme visuelle, sonore ou d’objets physiques. Le numérique déborde donc les seules technologies informatiques. Les humanités numériques renvoient alors à la rencontre des sciences et technologies informatiques et des sciences humaines et sociales.\fg{}, \cite{vinckDefinitionHumanitesNumeriques2016}.}. Les humanités numériques renvoient alors à la rencontre des ces disciplines.  

Les données produites, dans le cadre de ce projet de recherche, pour pouvoir être (ré)utilisées doivent répondre aux principes \acrshort{fair} (\acrlong{fair}). En français, ces principes \acrshort{fair} peuvent se traduire ainsi : Faciles à (re)trouver, Accessibles, Interopérables et Réutilisables. C’est une question qui a gagné en importance au cours des dix dernières années. L’idée est ainsi née de créer uns structure globale pour la publication professionnelle des données et des découvertes et pour l’échange et la réutilisation de celles-ci. En 2016, les \og{}The FAIR Guiding Principles for scientific data management and stewardship\fg{} sont publiés dans \textit{Scientific Data} \footnote{\cite{aalbersbergFAIRGuidingPrinciples2016}.}. Ces principes donnent une définition détaillée de \textit{Findability}, \textit{Accessibility}, \textit{Interoperability} et \textit{Reusability}. Ces principes doivent être applicables à la fois pour l’humain et la machine. En effet, on compte de plus en plus sur cette dernière pour gérer des données, en raison notamment de l’augmentation de leur volume, de leur complexité et de la vitesse à laquelle elles sont créées. Trouver une donnée est essentiel à tout ce processus et devrait être le plus facile possible d’où la position première de \textit{Findable}. Une fois les données trouvées, elles se doivent d’être accessibles mais cet accès peut se faire selon différentes modalités et peut parfois nécessiter certaines autorisations. Il est également nécessaire que les données auxquelles on a ainsi pu accéder puissent fonctionner les unes avec les autres. Elles doivent aussi être en mesure d’interagir avec des applications permettant leur analyse, leur stockage, leur traitement ou leur visualisation. Enfin, tout cela doit pouvoir être réutilisé par une autre personne que l’auteur : cela constitue l’objectif de l’ensemble de ces principes. Ces définitions sont l’occasion de poser un cadre permettant de mieux saisir les différents enjeux qui sont attachés aux données de la recherche. Parmi ces enjeux, celui qui a été au cœur de mon stage est la réutilisation, à des fins de médiation et de valorisation, des données produites et utilisées, ici dans le cadre d'un projet de recherche en histoire de l'astronomie.\\

Lors de précédents stages réalisés par des étudiants de master \acrshort{tnah} à l'Observatoire de Paris, l'attention s'est portée sur la médiation des données de la recherche \footnote{Voir par exemple, \cite{albouyMediationDonneesRecherche2019}.}. Dans le cadre de mon stage, il s'agit d'avantage de valorisation de ces données à travers une exposition en ligne intitulée, \textit{Medieval skies under(book)cover} \footnote{Une première version de l'exposition \href{https://alfa-exhibition.herokuapp.com}{\textit{Medieval skies under(book)cover}} est disponible.}. Cette exposition développée dans le cadre du projet \acrshort{alfa} (\acrlong{alfa}) a pour sujet l'astronomie alphonsine telle qu'elle se manifeste dans des manuscrits produits entre le XIIIe et le XVIe siècle. L'objectif principal de mon stage était la scénographie de cette exposition, c'est à dire la transcription du propos scientifique de l’exposition en un parcours de visite. Par cette mise en scène, les visiteurs doivent être transportés dans l'univers de l'astronomie alphonsine. Qu'est-ce que le choix d'une médiation numérique apporte à l'exposition ? Quels sont les apports, les enjeux et les limites qui découlent de ce choix ? Comment concevoir une plateforme d'exposition qui soit en mesure de satisfaire les exigences techniques et scientifiques du projet tout en garantissant une expérience  qualitative et enrichissante pour les visiteurs ? \\

Ce mémoire, réalisé en vue de la validation du master \acrshort{tnah} de l'\acrlong{enc}, s'efforcera donc d'apporter des éléments de réponses aux questions ci-dessus. Ainsi, une première partie présentera le projet, les travaux des chercheurs et les données produites dans ce cadre en amont de leur mise en valeur par le biais d'une exposition. Cet objet, complexe, sera également défini. La seconde partie sera consacrée à la mise en avant d'éléments directement liés à la scénographie : les parcours de l'exposition seront mis en lumière de même que l'analyse nécessaire des besoins du public. L'importance de l'existence d'un dialogue constant entre ingénierie et recherche sera aussi examinée. Enfin, la dernière partie portera sur le développement effectif de l'interface d'exposition en montrant l'utilité de l'emploi de ressources \acrshort{iiif}, en signalant la volonté de réaliser une exposition à l'architecture modulaire avant de finir par des considérations sur le futur de cette exposition virtuelle.
