\part{Scénographie d'une exposition virtuelle}
	
	\chapter{Les parcours de l’exposition}
	Pour débuter, il convient de dire quels sont les différents parcours retenus pour naviguer au sein de l'exposition puis d'évoquer la manière dont les liens entre lesdits parcours sont envisagés.
	
	\section{Présentation des parcours}
	\subsection{Parcours patrimoine - \textit{Sensing}}
	Les parcours relatifs aux aspects patrimoniaux sont au nombre de 7, ils sont regroupés sous le nom de \textit{Sensing} au sein de l'exposition. Ces différents parcours se concentrent donc la matérialité des manuscrits (les manuscrits que l'on touche), sur des détails d'ordre visuel (les manuscrits que l'on voit) ou intellectuel (les manuscrits que l'on lit). 
	
	\subsubsection{\textit{Touching the manuscript: the material dimension of the corpus}}
	C'est la dimension matérielle des manuscrits qui est mise en avant dans ce parcours : la reliure ou bien la composition en cahier par exemple. L'un des buts est de souligner la complexité du livre manuscrit, en comparaison avec celle d'un livre imprimé. Il s'agit aussi de montrer comment les figures historiques présentent dans l'exposition utilisaient cette complexité au service de leurs travaux. 
	
	\subsubsection{\textit{Seeing manuscript: the visual dimension of the corpus}}
	Ici ce sont les principaux éléments visuels qui sont mis en valeur : mise en page, enluminures, décorations et illustrations, dispositif de lisibilité (ponctuation), diagrammes ou encore format des tables. C’est la richesse visuelle de ces documents qui est présentée aux visiteurs.

	\subsubsection{\textit{Student manuscript}}
	Produit dans un milieu universitaire par des étudiants, ces manuscrits peuvent être considérés comme des cahiers de cours. Bien que visuellement moins attractifs que d'autres au premier regard, ces manuscrits sont plus vivants avec notamment de nombreuses notes en marge indiquant des interventions des étudiants. Ils renseignent notamment sur l'organisation du cursus universitaire et sur les associations faites entre différentes disciplines : ainsi l'astronomie est souvent associée à d’autres disciplines mathématiques telles que l'optique, la géométrie ou l'arithmétique mais aussi parfois à des textes de médecine voire même de théologie. Certains de ces manuscrits d’étudiants ont appartenu à des acteurs majeurs dans l’histoire de l’astronomie alphonsine comme Conrad Heingarter par exemple.

	\subsubsection{\textit{University manuscript}}
	Ces manuscrits sont majoritairement produits au XVe siècle dans le milieu des universités. Le plus souvent, ils sont produits dans le cadre de la \textit{peccia} qui est un système qui permettait à plusieurs copistes de recopier en même temps un ouvrage, ce système s'est développé avec la nécessité de fournir aux étudiants les textes indispensables à leurs études, ces manuscrits constituent ainsi des sortes de manuels. Certains de ces manuscrits peuvent aussi être réalisés dans des ateliers spécialisés. Selon leur contexte de fabrication, ils peuvent être plus ou moins précieux. 

	\subsubsection{\textit{Presentation manuscript}}
	Ces manuscrits sont, en général, produits par des ateliers spécialisés où l'on trouve à la fois des copistes, des miniaturistes, des enlumineurs, etc.. Il en existe deux principales sortes. Les premiers sont directement produits par des cours pour en assurer le prestige intellectuel, pour la tradition alphonsine c'est le cas de la cour d'Alphonse X ou de celle de Charles V \footnote{Né en 1338 et mort en 1380, il est roi de France à partie de 1364.}. Ils peuvent faire l'objet de présents diplomatiques ou de prise de guerre. Les autre sont produits par des savants qui les destinent à des personnages riches et puissants susceptibles de leur accorder leur protection. 

    Ce sont de très beaux manuscrits, d’une grande valeur patrimoniale, qui permettent de mieux saisir le contexte dans lequel se développe alors l'astronomie qui n'est pas restreinte aux seuls cercles scientifiques mais connaît un développement social et culturel bien plus large. Là encore, certains des manuscrits présentés dans l'exposition ont appartenu à des figures historiques d'importance première pour l'astronomie alphonsine. 

	\subsubsection{\textit{Toolbox manuscript}}
	Ces manuscrits qui constituent la quatrième grande catégorie de manuscrits présents dans l'exposition sont des documents véritablement issus de la pratique de l'astrologie. En effet, ils sont produits par des astrologues lorsque ceux-ci se consacrent à leur activité et sont composés de leur outils de calcul. Certains sont spécialisés dans une question astronomique ou dans un outil tandis que d'autres ont vocation à être plus généralistes et l'on pourra y trouve des textes, diagrammes ou tables portant sur différents sujets. 
	
	\subsubsection{\textit{Exceptional manuscript}}
	Si tous les manuscrits présentés dans cette exposition sont à leur manière exceptionnels, ceux de cette catégorie le sont encore davantage. Certains ont une organisation codicologique particulière, c'est le cas du \textit{batbook}, très peu de manuscrits de ce type subsistent encore, ce qui les rend encore plus précieux. D'autres font figure d'exception par ce qu'ils contiennent : uniquement des instruments, une "histoire" de l'astronomie par les textes ou encore des autographes de Regiomontanus, Jean de Saxe ou Jean de Murs.

	\subsection{Parcours historique - \textit{Learning}}
	On peut distinguer plusieurs étapes dans le développement de l'astronomie alphonsine à travers d'abord des origines en al-Andalus en lien avec les traditions arabes, vient ensuite un moment parisien avec le passage au latin. Après cela, le corpus parisien a voyagé ailleurs en Europe où de nouveaux textes ont vu le jour. L'ultime étape intervient à la fin de la période et correspond à l'amorce d'une transition vers le monde de l'imprimé. 

	\subsubsection{\textit{Royal origins in al-Andalus}}
	En se plaçant à la cour d'Alphonse X, ce parcours vise à mettre en lumière les origines de l'astronomie alphonsine et ses liens avec l'astronomie arabe. Ce parcours s'appuie donc en majorité sur des manuscrits produits à la cour d'Alphonse X mais également sur d'autres s'ils contiennent des éléments issus de la tradition des tables de Tolède, d’Al-Khwarizmi ou d’Al-Battani.

	\subsubsection{\textit{Paris University}}
	Dans ce parcours, l'accent est mis sur l'étape parisienne de l'histoire de l'astronomie alphonsine et sur les œuvres de Jean Vimond, Jean de Lignères, Jean de Saxe et Jean de Murs qui ont participé à une première diffusion latine de l’astronomie alphonsine.
	
	\subsubsection{\textit{Fostering astronomical activities around Europe}}
	Le but de ce parcours est de montrer comment l'astronomie alphonsine s'est étendue à d'autres horizons européens en proposant aux visiteurs de s'intéresser à des manuscrits et des œuvres produits en Angleterre, en Italie ou en Europe centrale.  
	
	\subsubsection{\textit{Transition to the printing world}}
	Dans ce dernier parcours, il s'agit de se placer à la fin du XVe siècle quand des cercles humanistes ont assuré les premières impressions de textes issus du corpus alphonsins, Bianchini, Regiomontanus ou Peurbach par exemple.
	
	\subsection{Parcours scientifique - \textit{Understanding}}
	Les parcours de ce thème ont pour objectif de familiariser les visiteurs avec le contenu scientifique des manuscrits alphonsins. Par contenu scientifique, il faut entendre ici les phénomènes astronomiques étudiés aussi bien que les outils mathématiques indispensables à ces études. 
	
	\subsubsection{\textit{Spherical astronony}}
	C'est un type d'astronomie qui s’intéresse au lever et coucher des astres, aux différentes longueurs du jour et de la nuit selon la période de l’année et la latitude, aux ombres etc. C’est une partie essentielle de l’astronomie, elle permet notamment de déterminer l’ascendant, point clé de toutes prédictions astrologiques. L'objectif de ce parcours est de fournir aux visiteurs des notions de base sur cette astronomie en s'appuyant sur les manuscrits exposés et les outils qu'ils renferment. 

	\subsubsection{\textit{Planetary astronony}}
	À l'image du précédent, ce parcours donnera des notions de bases sur un autre type d'astronomie qui étudie la manière dont les mouvements en longitude et en latitude des planètes étaient modélisés et calculés. L'objectif est ici de montrer comment les textes, diagrammes et tables sont utilisés dans ce cadre.
	
	\subsubsection{\textit{Eclipse}}
	Là encore, en s'appuyant sur les manuscrits, ce parcours accompagnera les visiteurs verts la découverte de notions de base au sujet des éclipses. C'est le dernier parcours lié à un type d'astronomie en particulier.
	
	\subsubsection{\textit{Diagrams and instruments}}
	Ce parcours, comme les suivants, se concentre sur un outil intellectuel utilisé par les astronomes médiévaux pour étudier les phénomènes astronomiques plutôt qu'à un phénomène en particulier. Ici, on propose aux visiteurs de s'intéresser aux diagrammes et la manière dont ils sont utilisés. Certains de ces diagrammes sont en fait des instruments, parfois même des instruments de calculs.

	\subsubsection{\textit{Astronomical tables}}
	De la même manière, on souhaite ici attirer l'attention des visiteurs sur les tables et sur certaines des valeurs significatives qu’elles contiennent. Un autre point d'attention est le type de nombres utilisés.

	\subsubsection{\textit{Texts}}
	Dans ce dernier parcours lié à un type d'instrument, on présente aux visiteurs les différents types de textes présents dans le corpus alphonsin et la manière dont ils contribuent à mettre en mouvement et en relation les tables et les diagrammes.

	\subsubsection{\textit{Other disciplines}}
	Ultime parcours de cette section dédiée aux éléments scientifiques, ce parcours s'attache à montrer les liens qui existent, notamment dans le milieu universitaire, entre l'astronomie et d'autres sciences : autres disciplines mathématiques, médecine ou théologie.
	
	\subsection{Parcours manuscrits - \textit{Reading}}
	L'intégralité des manuscrits ayant été présentée dans le chapitre 2, les parcours relatifs à chacun d'entre eux ne seront pas détaillés une nouvelle fois. Il importe simplement de dire qu'un parcours est prévu pour chacun des quinze manuscrits, ces parcours permettront aux visiteurs de saisir la richesse qu'ils contiennent.

	\section{Matrice et \textit{deep linking}}
    Tous les parcours précédemment présentés peuvent fonctionner individuellement mais ils ne sont pas dépourvus de lien les uns par rapports aux autres. En effet, il est aisé de constater que ceux relatifs à un même thème peuvent être considérés comme les différentes parties d’un même parcours bien plus étendu mais d’un thème à l’autre il existe aussi des ponts notamment rendus possibles par la richesse de contenu des manuscrits : un manuscrit appartient souvent à plusieurs parcours. De plus, certains \textit{attention points} peuvent, eux aussi, être communs à différents parcours. Pour mettre cela un peu plus au clair, un tableau de ces \textit{attention points} communs \footnote{Ce tableau est disponible en annexe \ref{APcommuns}, à la page \pageref{APcommuns}.} a été réalisé pour venir compléter celui déjà produit par les chercheurs pour garder une trace de la répartition des \textit{attention points} au sein des divers parcours. Dans ce tableau, un code couleur simple est utilisé pour déterminer dans combien de parcours un \textit{attention point} peut être rencontré et des liens renvoient vers l’exacte position de cet \textit{attention point} dans chacun des parcours concernés. Ces liens précis sont essentiels pour la construction de l’exposition telle que l’équipe la souhaitait. En effet, l’idée n’est pas ici d’imposer aux visiteurs un unique sens de progression déjà tracé au sein de l’exposition. Au contraire, l’utilisateur, bien que guidé, est libre d’aller et venir entre les différents thèmes en commençant où bon lui semble. 
    
    Néanmoins, pour que chacun puisse prendre part à une visite complète de l’exposition, des pistes de poursuite de la navigation peuvent être suggérées. Ainsi, le visiteur peut être invité à une découverte plus approfondie des questions scientifiques après avoir visité le parcours ayant pour objet les manuscrits d’étudiants. Par ailleurs, pendant sa navigation au sein d’un parcours, le visiteur aura également la possibilité de changer de direction. Par exemple, si dans le parcours dédié aux tables astronomiques, un \textit{attention point} fait mention du manuscrit \acrshort{bnf} Mélanges Colbert 60, le visiteur aura la possibilité de bifurquer vers le parcours entièrement consacré à cet unique. Inversement, un visiteur qui serait en train de parcourir ce même manuscrit serait en mesure de se rendre vers le parcours relatif au moment parisien de l’astronomie alphonsine via un \textit{attention point} en faisant mention. 
    
    Certains des parcours envisagés par les chercheurs sont très longs et le temps à y consacrer dépasse nettement la quinzaine de minutes fixée au départ. Pour pallier les éventuels problèmes que cela pourrait engendrer, la solution suivante a été proposée : créer deux versions de ces parcours, une complète et une courte. Cette version raccourcie pourra satisfaire des visiteurs pressés et/ou novices en leur proposant de ne s’attarder que sur les pièces majeures du parcours. Pour ceux qui le souhaitent, à la fin de cette version réduite, l’accès à la version longue sera permis. Dans le même ordre d’idée, l’inclusion de \textit{must see} est également envisagée en s’appropriant l’approche de \textit{gateway object} décrite par les équipes du British Museum \footnote{\cite{battyObjectFocusedText2016}.} pour des expositions \textit{in situ} et en l’adaptant à une visite en ligne.  Il s’agit de désigner quelques objets représentatifs de l’ensemble et attractifs pour le visiteur qui pourra s’intéresser à chacun d’eux dans l’ordre de son choix et accéder à la fois aux principaux objets en tant que tels mais également, à travers eux, aux principales clés de compréhension de l’astronomie alphonsine. Qu’il s’agisse de la sélection de \textit{must see} ou de la création de certains parcours dans une version plus courte, le travail doit encore être poursuivi mais l’idée est là et des tests ont pu être réalisés pour leur mise en œuvre concrète sur le site de l’exposition. 
    
    En suivant toujours cette même logique de ne pas imposer d’unique manière d’appréhender cette exposition aux visiteurs, ce qui a en revanche déjà été implémenté est un affichage des parcours relatifs à un thème dans un ordre aléatoire sur la page dédiée. Cela souligne bien qu’il n’est pas nécessaire d’avoir visité un parcours pour en comprendre un autre, chacun peut être positionné où le visiteur le souhaite dans sa visite. Néanmoins, certains parcours sont plus accessibles que d’autres, ainsi pour un visiteur n’ayant peu ou pas de connaissances en astronomie, commencer par les parcours à caractères scientifiques pourrait rendre la progression plus difficile voire la stopper par perte d’intérêt devant la complexité de certaines notions. C’est pour cela que sans rien imposer, ces parcours arrivent en troisième position dans la barre de navigation : si le visiteur suit la progression ainsi suggérée, il aura déjà pu se familiariser avec les documents avant de s’engager dans les parcours les plus techniques. 

    Ainsi, cette exposition est conçue pour être une sorte de matrice ouvrant des possibilités multiples aux visiteurs qui peuvent choisir de se laisser pleinement guider par les chercheurs ou se laisser porter par leur propre curiosité. 

	
	\chapter{Analyse des besoins des publics}
	Vient à présent le temps de se consacrer de façon plus détaillée aux attentes des publics en précisant d'abord qui ils sont avant de s'engager dans quelques réflexions sur l'expérience qui doit être celle des utilisateurs et sur la fréquentation des expositions. 
	
	\section{Public(s) ciblé(s)}
    Certaines réflexions avaient déjà été portées en amont et quatre catégories de public avaient ainsi pu être déterminées : étudiants, chercheurs, professionnels du livre et une dernière catégorie bien plus large faisant référence à des visiteurs variés. Une première étape a été d’essayer d’établir quels pouvaient être les intérêts et les besoins de tous ces potentiels visiteurs. 
    Ainsi, pour les différents publics envisagés, il est apparu nécessaire de chercher à savoir quelles étaient leurs activités principales, où est-ce qu’ils travaillaient, par quels moyens seraient-ils être en mesure d’entendre parler de l’exposition, quelles pourraient être leurs attentes ? Il était également important de songer dès lors à des idées pour que lesdites attentes soient satisfaites. 

    \subsubsection{Les professionnels du livre}
    Viennent premièrement les professionnels du livre. On suppose qu’ils travaillent dans des bibliothèques ou des institutions patrimoniales et que c’est probablement sur leur lieu de travail qu’ils auront connaissance de l’exposition. Ils sont susceptibles d’être demandeurs d’informations bibliographiques précises, ce qui peut être fait en indiquant des ressources extérieures telles que les notices des manuscrits publiées sur les site web des institutions de conservation ou encore les pages wikipedia des manuscrits lorsqu’elles auront fait l’objet d’une mise en ligne. Un autre intérêts pour eux peut résider dans la mise en valeur de leurs propres collections. Néanmoins, cela ne constitue pas nécessairement une motivation pour visiter l’exposition mais sans doute plutôt pour y prendre part. 
 
    \subsubsection{Les chercheurs}
    À l’instar des précédents, les chercheurs constituent un public de professionnels ayant déjà acquis des connaissances sur au minimum une des principales thématiques de l’exposition : considérons donc ici les chercheurs dans les domaines de l’histoire des sciences ou de la codicologie par exemple. Leurs attentes sont donc plutôt portées vers des explications détaillées accompagnées de ressources extérieures permettant d’aller plus dans le détail encore. Si le renvoi vers des ressources extérieures est tout à fait envisageable, ce n’est pas forcément le cas des explications complexes puisque cela rendrait l’accès beaucoup plus difficile à d’autres types de publics. Or, le but de cette exposition étant de valoriser des manuscrits astronomiques, il semble essentiel que l’audience puisse être la plus large possible jusqu’à toucher des personnes tout à fait novices sur les thématiques proposées. 

    Ainsi, et également pour satisfaire d’autres besoins de ce projet, en particulier le temps à passer sur chacun des parcours, le choix a été fait de se focaliser sur une audience large qui chercherait à découvrir l’astronomie alphonsine plutôt que de s’adresser vraiment à des spécialistes. En effet, dès la conception de l’exposition l’idée était plutôt d’avoir des textes simples et idéalement assez courts permettant de saisir les éléments essentiels et non d’entrer dans des détails peu accessibles à une audience non spécialiste. Néanmoins, les spécialistes ne sont pas exclus. En effet, l’ajout d’une section bibliographique sur le site web de l’exposition autorise un accès à des informations bien plus précises et à des ressources extérieures apportant plus de détails. De plus, l’idée étant, \textit{in fine}, d’avoir à la fois des versions longues et des raccourcis, au moins pour les parcours les plus longs, un public spécialisé pourra également être davantage intéressé par ces versions longues.

    \subsubsection{Les étudiants}
    Au sein de cette audience plus large finalement désignée comme public cible, il est possible de détacher un groupe particulier : les étudiants qui sont un public très éclectique, demandeur à la fois de contenu éducatif avec des exemples concrets et d’éléments plus ludiques. Cela en fait un public tout à fait intéressant pour une exposition en ligne. En effet, l’aspect interactif et immersif est propice à capter l’attention et à la maintenir plus longtemps. Par ailleurs, une telle exposition peut potentiellement être utilisée comme un support d’illustration par des professeurs ou parents dans le cadre du parcours scolaire. 

    \subsubsection{Un large public}
    En choisissant un public de visiteurs variés plutôt qu’un groupe précis comme public cible, l’équipe en charge de l’exposition est consciente qu’il sera difficile de répondre à des attentes aussi variées que le sont les visiteurs. Néanmoins, il est possible de mettre en avant certains souhaits communs, à commencer par la volonté de découvrir et d’en apprendre davantage sur l’astronomie ou sur les manuscrits. Ce public peut également être tout simplement désireux de porter un regard nouveau sur le Moyen Âge.  À l’image des étudiants, ce public novice peut aussi être friand et demandeur d’une dimension ludique. Aussi, une section de jeux a été envisagée, bien que celle-ci ne soit pas encore implémentée. 
    Néanmoins, face à un public si large, il n’est pas possible de cerner parfaitement les besoins et par conséquent de satisfaire les attentes de tous les éventuels visiteurs. Par ailleurs, si l’on veut vraiment inclure ce public le plus large possible, il est essentiel de garder en tête que si le numérique peut attirer, il peut aussi représenter un frein pour d’autres. Un plan du site ou bien un guide d’utilisation mériteraient donc d’être ajoutés. 

	\section{Expérience utilisateur}
	Pour répondre au mieux aux diverses attentes précédemment évoquées, des réflexions sur l’\textit{UX design} peuvent se révéler très utiles. 				\og{} L’expérience utilisateur est une forme spécifique d’expérience humaine qui naît de l’interaction avec une technologie ou un service. Le design UX a pour objectif de rendre cette expérience la plus positive possible en la pensant avant le produit\fg{} \footnote{\cite{gronierMethodesDesignUX2017a}.}. L ‘utilisateur est donc au cœur du processus de développement. Le produit en tant que tel ne se suffit plus à lui-même et l’expérience qui sera celle de l’utilisateur a toute son importance. Pour définir au mieux, les besoins de ces futurs utilisateurs, la méthode des personas peut-être utilisée. C’est un outil qui peut être défini ainsi : 

    \begin{quote}
	    Les personas [permettent] de produire des connaissances sur l’utilisateur, connaissances pouvant aider les concepteurs à développer des fonctionnalités utiles à la conception d’applications informatiques innovantes \footnote{\cite{brangierEffetsPersonasContraintes2012}.}. 
    \end{quote} 

    C’est un outil qui est de plus en plus utilisé et intégré dans les phases de conception d’objets numériques. S’il présente de nombreux avantages et permet de considérer de manière plus importante les utilisateurs, c’est un outils imparfait qui s’appuie en partie sur des idées préconçues sur le(s) public(s) ciblé(s). 


    Cette expérience utilisateur n’est pas qu’une question de rendu esthétique final, il s’agit aussi de prendre en compte les exigences du public en terme de navigation. Comme cela a été mentionné, le public ciblé cherche à découvrir de nouvelles choses mais sans que cela ne paraisse rébarbatif. Pour cela, la navigation se doit d’être fluide et intuitive. De plus, le temps passé sur chacun des parcours ne doit pas excéder une quinzaine de minutes afin de maintenir les visiteurs attentifs et captivés. L’utilisateur doit constamment être engagé et se sentir maître de sa déambulation au sein de l’exposition : il est le narrateur de l’histoire qu’il visualise. Les personnes en charge du développement du site web se doivent donc prendre cela en considération et faire en sorte que le visiteur puisse avoir accès à une véritable expérience. Les enjeux doivent être clairs et s’ils ne le sont pas, ils doivent être accompagnés d’explication. : cela participe au confort des visiteurs qui constitue l’un des points essentiels des réflexions sur l’\textit{UX design}. Il en va de même pour les éléments de navigation qui ne seraient pas intuitifs. Un plan du site peut s’avérer être une addition non négligeable pour permettre aux visiteurs d’avoir une idée d’ensemble de ce qui leur est proposé au sein de l’exposition. Quoi qu’il en soit, l’image doit être première au sein de l’exposition : elle doit faire sens et séduire le visiteur.

    \section{Fréquentation des expositions}
    L’exposition est une forme de médiation qui rencontre le plus souvent son public. En effet, les expositions à succès ne sont pas des faits rares, notamment dans le monde muséal. Les expositions conçues par les équipes de grandes bibliothèques telles que la \acrshort{bnf} connaissent aussi un vif succès et dans le cas de la \acrshort{bnf}, les documents mis en valeur peuvent être similaires à ceux mis en scène dans l’exposition \textit{Medieval skies under(book)cover}, cela prouve l’existence d’un public pour ce type d’opération de valorisation. 

    Comme le rapporte le Ministère de la Culture le 26 juin 2020, la pandémie de Covid-19 a marqué un changement important pour la fréquentation en ligne des collections patrimoniales. En effet, qu’il s’agisse des collections elles-mêmes ou des pages qui recensent ces mêmes collections, les chiffres ont été grandement multipliés : par exemple, la page du site Musées \footnote{\cite{ministeredelacultureetdelacommunicationChiffresClesStatistiques2021}.} où sont indiquées les expositions virtuelles a vu sa fréquentation augmenter de 2000\%. Le Ministère indique aussi que 
    \begin{quote}
    Même s’il est prévisible que ces chiffres baisseront après la réouverture des musées, certains internautes, séduits par la qualité des contenus, continueront de fréquenter les sites Internet des musées. 
    \end{quote}
    Si la note vaut pour les musées, une tendance similaire est observable pour la \acrshort{bnf}. En effet, en 2021, la fréquentation de Gallica a diminué de 2\% par rapport à 2020 mais en comparaison à 2019, elle est supérieure de 19\%. Cela montre bien que même si le contexte du confinement était particulier, certaines habitudes prises se maintiennent notamment en terme de consommation de la culture. C’est un paramètre à prendre en compte lors du développement d’une exposition en ligne. Ainsi, on peut supposer que le succès de l’exposition conçue pour le projet \acrshort{alfa} sera plus important que ce qu’il aurait pu être il y a quelques années du fait de ces nouvelles habitudes. C’est aussi l’une des raisons qui peut justifier de s’adresser à un large public pour cette mise en valeur de manuscrits astronomiques. Toutefois, pour garder vivante cette consommation en ligne, les données proposées aux visiteurs doivent l’être selon des modalités qui en plus de les mettre en valeur les rendent attractives aux visiteurs.
	
	\chapter{Gestion de projet}
	Parmi les missions confiées au cours de ce stage figurent des tâches relatives au suivi et à la gestion du projet au cours de différentes phases de développement. Seront évoquées tour à tour la question du dialogue avec les chercheurs et  celle de la rédaction de documentation.
	
	\section{Dialogue avec les chercheurs}
	\subsection{Intermédiaire entre ingénierie et recherche}
	Dans le cadre d'un projet de recherche, le dialogue est primordial pour coordonner les avancées et répartir le travail à effectuer. Ici, le dialogue doit aussi se faire entre les chercheurs et les ingénieurs, chacun est engagé dans le projet mais avec des missions à la fois différentes et complémentaires. Le rôle qui a été le mien était celui d'intermédiaire. En effet, il s'agit d'identifier, de comprendre et d'analyser les besoins des chercheurs tout en ayant des connaissances de la réalité du terrain qui permettent de mettre en perspective ces besoins et de se questionner sur leur faisabilité. C'est aussi l'occasion d'avoir un point de vue plus englobant prenant en compte les différents besoins et positions de chacun. Il faut donc faire preuve d'analyse et de synthèse pour garder à l'esprit les objectifs du projet et mettre en œuvre les solutions qui semblent les plus adaptées pour y parvenir. 
	
	Si recherche et ingénierie sont souvent perçues comme deux disciplines que l'on ne peut réunir, les humanités numériques, dont la place est grandissante, sont une preuve qu'un dialogue est possible et surtout bénéfique à l'un et l'autre de ces milieux. Cette dichotomie est assez caractéristique du paysage universitaire français \footnote{\cite{foucherAtlasInfluenceFrancaise2013}, p. 44}. Le développement de rapprochements entre ces deux sphères permet aux ingénieurs de mettre leurs compétences techniques au service des sciences humaines tandis que les chercheurs peuvent bénéficier d'outils qui leur étaient jusqu'alors inaccessibles mais qui se révèlent utiles pour leur travaux. Les possibilités permises et horizons ouverts ne cessent de croître par la présence plus visible de projets interdisciplinaires. C'est le cas du projet \acrshort{alfa} où histoire des sciences et valorisation numérique sont intrinsèquement liées. La présence sur place d'une \textit{\acrshort{dh} team} souligne cela. Tous les lundis un point hebdomadaire est fait sur le travail réalisé au cours de la semaine passée et sur les objectifs pour la semaine à venir pour les membres de cette petite équipe. Ces réunions ont lieu avec Matthieu Husson, chef du projet de recherche \acrshort{alfa}, et Ségolène Albouy, cheffe de projets numériques. Ces rendez-vous sont aussi l'occasion de discuter des idées et ainsi statuer sur celles qui sont à explorer et celles qui sont à exclure. 
	
	\subsection{Séminaires DH}
	Des temps d'échange avec l'ensemble de l'équipe sont également prévus notamment par le biais de séminaires internes à l'équipe et ayant pour objet les \acrlong{dh}. Lors de ces séminaires, des outils informatiques peuvent être présentés mais ils peuvent aussi être l'occasion pour l'équipe en charge de l'ingénierie de présenter son travail aux chercheurs. Pour des projets réellement communs comme cette exposition, ce séminaire est un moment important d'échange et de réflexion. Au cours de mon stage, trois séminaires \acrshort{dh} ont eu lieu : le premier, animé par Ségolène Albouy, présentant le protocole \acrshort{iiif} et les deux autres relatifs à l'exposition et que j'ai pu préparer et présenter, accompagnée de la cheffe de projet numérique. 
	
	Le premier a eu lieu en mai et le but était alors de faire un point sur les ressources à disposition au début du stage et de poser les questions qui étaient les miennes à propos de ces dernières, de présenter à l'ensemble de l'équipe, chercheurs et ingénieurs, des idées de scénographie et de proposer d'utiliser Exhibit plutôt que Storiies Editor. Les riches discussions survenues au cours de ce séminaire ont grandement participé aux réflexions alors en cours relatives à la scénographie et à l'expérience qui doit être celle des visiteurs. L'éventuelle création de versions écourtées pour les parcours les plus longs a notamment été évoquée lors de ce séminaire, de même que la question du titre à donner à l'exposition. 
	
	Le second a, quant à lui, eu lieu en juillet donc à un stade plus avancé du stage. La première partie a été consacrée à une présentation du site de l'exposition tel qu'il était alors : en construction. Cela a permis de réfléchir ensemble aux ajouts à faire et aux éléments à améliorer. De plus, ce séminaire a donné l'occasion de faire un point sur les manuscrits sélectionnés pour apparaître dans l'exposition et sur les modalités de communication pour ceux qui n'étaient pas numérisés ou alors dont la numérisation n'était pas libre de droit. La deuxième partie du séminaire s'est déroulée sous forme de \textit{workshop} pour permettre aux chercheurs de prendre en main l'outil Exhibit avec la possibilité d'être guidés et d'obtenir une réponse immédiate à leurs éventuelles questions. La préparation de ce séminaire a donné lieu à la rédaction de documentation relative à Exhibit pour également fournir aux chercheurs un document écrit auquel se référer. Celui-ci a pu être complété suite aux questions posées au cours du \textit{workshop} \footnote{Les présentations utilisées en support de ces deux séminaires sont disponibles en annexe \ref{Séminaires}, à la page \pageref{Séminaires}.}.

	\section{Rédaction de documentation}
	\subsection{À destination des chercheurs}
	Comme mentionné juste au dessus, au cours de mon stage j'ai eu l'occasion de rédiger de la documentation adressée avant tout aux chercheurs. Le projet et l'équipe étant internationaux, l'ensemble des documents a été rédigé en anglais. La principale documentation à destination des chercheurs est un tutoriel pour réaliser des \textit{exhibits} grâce à l'outil du même nom \footnote{Ce document a été reproduit tel quel, donc en anglais, en annexe \ref{Tutoriel}, à la page \pageref{Tutoriel}.}. Ce document propose treize étapes qui doivent permettre de construire ou de modifier un parcours de l'exposition, selon les règles qui ont été discutées avec l'ensemble de l'équipe au cours du \textit{workshop}. Les treize étapes sont les suivantes : 
	\begin{enumerate}
    \item Trouver ou créer un \textit{exhibit}
    \item Définir les paramètres
    \item Ajouter un \textit{manifest} à l'\textit{exhibit}
    \item Rédiger une \textit{caption}
    \item Ajouter des questions
    \item Ajouter des \textit{pinpoints}
    \item Traduire et harmoniser l'utilisation de l'italique et des guillemets 
    \item Changer l'ordre des \textit{attention points} d'un parcours
    \item Trouver des titres
    \item Ajouter des liens dans les \textit{captions}
    \item Dupliquer un \textit{exhibit}
    \item Rédiger ou compléter les \textit{attention points}
    \item Conserver les liens 
    \end{enumerate}

    Toutes ces étapes ne sont pas obligatoires à chaque fois, cela dépend du parcours sur lequel le chercheur est en train de travailler. Si certains de ces éléments seront détaillés davantage dans le chapitre suivant, ce n'est pas le cas de tous et ceux dont il ne sera plus questions vont donc faire l'objet de quelques précisions dès maintenant. Les manuscrits exposés sont des manuscrits en latin, or on peut supposer que l'intégralité des visiteurs ne sera pas pleinement confortable avec la lecture de cette langue. Ainsi, toutes les citations latines doivent être traduites en respectant ces règles : la citation latine est retranscrite entre guillemets tandis que sa traduction est ajoutée entre parenthèses. Les titres sont quant à eux donnés en italique. Définir ces règles simples permet une harmonisation à l'échelle de l'exposition alors que différentes pratiques pouvaient être observées du fait des différentes mains ayant pris part à la rédaction des \textit{attention points}. Un autre point à expliciter est le numéro 8. Cela concerne notamment les parcours thématiques. En effet, à l'intérieur de ces parcours les \textit{attention points} ont été insérés dans les \textit{exhibits} manuscrit par manuscrit, ce qui n'est pas une solution optimale pour capter l'attention des visiteurs. Une progression thématique mélangeant les différents manuscrits est à privilégier. Concernant le point numéro 9, certains \textit{attention points} s'étaient vus attribuer des titres, là encore selon la personne qui a procédé à leur rédaction. C'est une bonne idée qui aiderait les visiteurs à se repérer dans l'exposition et trouver des titres pour les \textit{attention points} n'en disposant pas encore serait une manière d'améliorer les \textit{exhibits}. Enfin, pour certains manuscrits la totalité ou une partie des \textit{attention points} ne sont pas rédigés et il conviendrait donc d'y remédier.

    En complément de ce document, d'autres relatifs aux \textit{exhibits} ont été produits : un tableau récapitulatif de tous les \textit{attention points} communs à plusieurs parcours \footnote{Déjà évoqué,ce tableau récapitulatif est disponible en annexe \ref{APcommuns}, à la page \pageref{APcommuns}.} et un autre, qui, pour chaque figure historique mentionnée, indique le folio et donc l'\textit{exhibit} concerné en fournissant le lien, les liens vers les ressources telles que Biblissima et Wikipedia sont également indiqués.

	\subsection{À destination d'ingénieurs}
	Une documentation avec un caractère plus technique et donc davantage destinée à des ingénieurs a également été rédigée avec en premier lieu l'élaboration d'un document de spécifications. Ce document vise à lister le contenu des pages du site web, les liens qui devront être disponibles sur chacune de ces pages mais également les éléments à vérifier pour le fonctionnement. Ce document permet d'avoir une base des attentes à laquelle se référer dans toutes les étapes ultérieures du développement du site web. S'il peut aussi être utile aux chercheurs, ce document présente un véritable intérêt pour de futurs développeurs chargés de compléter le site de l'exposition en y ajoutant notamment les pages envisagées mais pas encore implémentées : ils auront alors accès aux informations nécessaires concernant la place de ces pages dans l'architecture globale du site web et le contenu qui doit être le leur \footnote{Ce document a été réalisé après avoir analysé les différentes ressources à disposition et avant d'entamer la phase de réalisation, il peut être consulté en annexe \ref{Spécifications}, à la page \pageref{Spécifications}}. 
	
	Outre cela, une documentation du code de l'application est également disponible sur le \textit{repository} du projet d'exposition, accessible sur le GitLab de l'Observatoire de Paris. Git est un outil de \textit{versionning} qui permet le développement informatique à plusieurs mains en travaillant sur différentes branches. La possibilité d'accéder à des versions antérieures du projet peut constituer en soi une forme de documentation. Néanmoins, la véritable documentation accessible via GitLab correspond aux Wikis qui ont été rédigés. Ils contiennent des informations relatives à l'installation de l'application mais pas seulement. Toutes sortes d'éléments y sont décrits qui doivent être en mesure de permettre à d'autres ingénieurs, plus tard dans ce même projet ou pour un autre projet, de comprendre l'arborescence des dossiers ou encore de savoir quels sont les changements à effectuer pour ajouter, supprimer ou modifier un élément. \\
	
	Cette seconde partie a permis de mettre en lumière des questionnements liés à la scénographie virtuelle de cette exposition en évoquant tour à tour la navigation entre les parcours et les besoins du public. Enfin, la question du dialogue entre recherche et ingénierie a été posée et on a montré combien ces discussions ont pu être constructives pour l'élaboration d'une mise en scène satisfaite des ciels médiévaux. 
	
