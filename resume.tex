\chapter{Résumé}
	\medskip
	Ce mémoire, réalisé dans le cadre du Master « Technologies numériques appliquées à l’histoire » de l’École nationale des chartes, a pour objet le stage effectué à l'Observatoire de Paris afin de mettre en œuvre la scénographie d'une exposition virtuelle de manuscrits astronomiques. Les phases de développement y sont explicitées, de l'analyse du contexte de recherche et des données produites à la mise en production d'une interface pour l'exposition \textit{Medieval skies under(book)cover} qui donne à découvrir ces manuscrits sous différents angles : codicologie, histoire des sciences et des textes ou encore pratiques scientifiques. L'expérience des visiteurs tient une place non négligeable dans les réflexions proposées. Le présent travail vise à observer et analyser les enjeux, les outils et les résultats de l'utilisation du numérique dans la valorisation d'objets patrimoniaux. \\
	
	\textbf{Mots-clés~:} exposition virtuelle ; astronomie alphonsine ; histoire des sciences ; Observatoire de Paris ; médiation culturelle en ligne ; expérience utilisateur. \\
	
	\textbf{Informations bibliographiques~:} Anaïs Mazoué, \textit{Medieval skies under(book)cover : scénographie d'une exposition de manuscrits astronomiques sur le web}, mémoire de master \og{}Technologies numériques appliquées à l'histoire\fg{}, dir. Lucence Ing, École nationale des chartes, 2022.
	