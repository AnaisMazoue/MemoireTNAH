\documentclass{article}
\usepackage[utf8]{inputenc}

\usepackage{minted} % colored source code

% configurer le document selon les normes de l'École
\usepackage[margin=2.5cm]{geometry} %marges
\usepackage{setspace} % espacement qui permet ensuite de définir un interligne
\onehalfspacing % interligne de 1.5
\setlength\parindent{1cm} % indentation des paragraphes à 1 cm

\title{\textit{Exhibit tutorial}}
\author{anais.mazoue }
\date{September 2022}

\begin{document}
	\section{Find or create an Exhibit}
    Work directly on the Exhibit link that you’ll find in the third column here. The links found in this column are the one that will be used to make the improvements. For some of the paths, some modifications have already been made but mostly to try and see what could be made. The easier way to work is probably to begin with the manuscripts' paths rather than the thematic ones. Indeed, once the first category is done, you’ll be able to simply copy and paste the attention points as they’ve been improved in the specific paths they are part of (you’ll maybe have to add captions or do divide them but at least a part of the work will already be done). 
    You can also create a new Exhibit here if you want to add a new path to the exhibition. 

	\section{Complete the settings}
    If you create a new Exhibit, you land directly on the settings page. Otherwise, on the edit page of an Exhibit that already exists, the settings can be accessed and modified by clicking on the small pen at the top right corner of the box that contains the title and the description. 
    Then you can:
    \begin{itemize}
    \item select the quiz template if it is not already done
    \item modify the title
    \item complete or modify the authors list (for some I didn’t know the last names so I just wrote the first name followed by a question mark…)
    \item write or complete the description of the path (I only translated what had been written here so something else might have to be written, my translations have to be corrected because I left some things in French when I didn't know the proper way to translate…). The length of the text is limited to 1000 characters(for example, the University manuscript path description is too long so it has to be shortened.
    \item write an optional “Quiz Complete Message”: it is possible to have links in the message so an idea could be to allow the visitors to go to another path through this message 
    \item the “Rights” section is optional but maybe the libraries where the manuscripts are kept should be mentioned here ? 
    \item keep the “Access” as it is with the “Public” selected 
    \item keep the “Allow Duplication” selected as well, at least for now that it might still be useful to create shortcuts but this would probably have to be unselected when all the work is done 
    \item select “Je ne suis pas un robot” and “I have read and agree…”
    \item click on the “Update” button
    \end{itemize}

    \section{Add a manifest to the Exhibit}
    When you create a new exhibit, you need to add one or more manifest(s) but you can add a manifest later as well. Either way, you can do that by clicking on the “+ Add Item” purple button at the bottom left corner of the page. A little window will open where you can add a IIIF Manifest URL and then click on the “Import” button to be able to use it in your exhibit. When you want to use a manifest to add a new attention point, you need to select it in that same window and click on the “Add to exhibit” button. When adding a new attention point, be careful to have the corresponding manifest selected. 

    \section{Add a caption}
    At the bottom of the edit page, you will find a “+” button that you’ll have to click on in order to add an item to the exhibit. When doing so, be careful to have the right manifest selected. You can then write a caption and edit it, you can choose your image in the right part of your screen. 
    For some of the items I created, there was no caption and/or no folio indication so there are some empty captions that you can fill and the images are not always the right one because if you don’t choose an image the image that will appear is either the first of the manifest or the one used in the previous item, that explains why some folios appear more than one time when there is an empty attention points so when completing an attention point don’t forget to add the correct folio if it is not already the case. 
    Divide the caption when needed
    Some of the attention points are too long and need to be divided. That also allows more precise zooms on the concerned folios.

    Example : ms BnF 7295A AP5
    This is the attention point as it was written. Since it is only one section in Exhibit, it can only be associated to a single folio with a single zoom. Dividing the caption in different sections allows to have as many zoomed images as there are sections so the zooms are more precise and more related to the text next to them. 

    In some cases, when an attention point is used in several paths, the entirety of the caption is not needed for all said paths and it would make the storytelling and navigation easier to only keep the part actually related to the current path.  

    Example : ms 7432 AP1
    Here dividing the caption allows for better zooms but it is also good to keep only the part one would be interested in when following a specific path. Indeed, this attention point is related to both seeing and touching paths but the part about the binding is only relevant for the touching one where the part about the modern numbering is all about seeing : a good solution is to divide the caption and keep what is needed for the path. For the path related to the whole manuscript, it is also better to divide because it allows a zoom on the binding and one on the numbering. 


    \section{Add questions/quizzes}
    Using the quizz template in Exhibit allows you to add questions to the captions. Indeed, as you would enter the text of the attention points, you can write a question and then add the possible answers. The small checkbox next to the correct one has to be selected. This is a possibility to add small games in the exhibition :)

    Highlight some parts of the attention points
    These highlights in bold letters are here for the audience to easily catch the more important information of the caption.

    \section{Add pinpoints}
    In order to help the visitor know where to look on a folio, pinpoints can be added. When you click on a caption, you are able to make modifications to it through the “Question” section, next to it is another one named “Pinpoints”. In this second section, you can simply double click anywhere on the image to create a pinpoint. You can add a label to your pinpoint if you want but it will not appear on the image. 

    \section{Translate and harmonize the use of italic and quotation marks}
    Some parts of the captions are quotes from the manuscripts so they are in Latin : some of them are accompanied by a translation, some of them are not. We can think that most of the audience won’t be latin experts so a translation for all the latin will be great. 
    The translations have to be harmonized : 
    \begin{itemize}
        \item write the Latin titles in italic letters with a capital letter at the beginning
        \item the Latin quotes between quotation marks in the main text
        \item the translation in parentheses for all the Latin except for the titles
    \end{itemize}

    \section{Change the order of the attention points in a path}
    In the thematic paths, the attention points are following each other manuscript by manuscript which is probably not the best way to do it. This change can be done with drag and drop in Exhibit to have a more coherent order in terms of storytelling. 

    Example : touching the manuscript 
    An idea could be to begin with all the attention points related to bindings and from that to go further inside the manuscripts to end with the added leaflets for example. 

    \section{Find titles}
    For now the titles are those used here. For some captions (manuscript 7295A) titles have also been added. It could be a good thing to have titles for the other captions, at least to “group” the attention that are about a similar subject. 

    Example : touching the manuscript 
    Here I grouped attention points related to the bindings of the manuscripts under a common title “Binding” (that title is not the best…). 

    \section{Add links in the captions}
    These links point the audience to the Exhibits related to the folios/manuscripts mentioned in the attention point. Links for a precise slide of the Exhibit are now available later so we will  be able to include those in the exhibition. The links will have to be added once the attention points are finished.  
    Links can be added in the “Quiz Complete Message” section of the settings too. 
    Since there is a different color for each of the main themes of the exhibition, an idea is to have a colored emoji with the link to help the visitors in their navigation on the website. 

    \section{Duplicate an exhibit}
    That is a functionality that might be useful when doing the shortcuts for the longer paths. Indeed it is possible to duplicate an exhibit and then in this duplicated version to delete the attention points that will not be part of the shortcut. 
    In order to duplicate an exhibit, one must click on the “Preview” button at the bottom of the edit page and then on the “Duplicate” button in the page that will be opened. After that, you can edit the duplicate exhibit as any other exhibit, beginning with the settings. 
    When all this work is done maybe allowing duplication won’t be something to keep. To change that, the one thing to do is to uncheck the “Allow Duplication” button in the settings. 

    \section{Complete the AP that are not written or not complete or indicate if they are not to be included in the end}
    For several manuscripts, there are empty attention points listed in the column named “AP to complete” of this document so they can be written. If it was decided not to include some of those attention points, it could be good to indicate it.

    \section{Keep the links}
    When you create a new exhibit, be careful to keep the edition link to be able to find and edit your exhibit later. The best practice would probably be to add a new line to this table for all new paths and to keep all the links and information in one place. 
\end{document}
